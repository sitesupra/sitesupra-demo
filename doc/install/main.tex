\documentclass[12pt]{article}

\usepackage{graphicx}
\usepackage{setspace}
\usepackage{datetime}
\usepackage{hyperref}
\usepackage{alltt}
\usepackage{tabularx}
% This fixes table break, but breaks small tables
%\usepackage{ltablex}
\usepackage{svnkw}
\usepackage[usenames,dvipsnames]{xcolor}

\svnidlong
{$HeadURL$}
{$LastChangedDate$}
{$LastChangedRevision$}
{$LastChangedBy$}

\newcommand{\todayYMD}{\the\year\twodigit\month\twodigit\day}
\newcommand{\vigClientBranding}{0}
\newcommand{\vigShowNotes}{1}
\newcommand{\vigClientName}{Client Name}
\newcommand{\vigClientNameShort}{client}
\newcommand{\vigProjectName}{SiteSupra 7}
\newcommand{\vigProjectNameShort}{supra7}
\newcommand{\vigAttn}{Person Name}
\newcommand{\vigDocumentRevision}{\svnrev}
\newcommand{\vigPackageName}{supra7}
%\newcommand{\vigPathToProject}{\textless path\_to\_project\textgreater}
\newcommand{\vigPathToProject}{/var/www}
\newcommand{\vigPathToSrc}{/src}
\newcommand{\vigPathToWebroot}{\vigPathToSrc/webroot}
\newcommand{\note}[1]{\ifthenelse{\vigShowNotes=1}{
\textbf{NOTE:} 
\textcolor{Brown}{
#1
}
}{}}
\newcommand{\todo}[1]{\ifthenelse{\vigShowNotes=1}{
\textbf{TODO:} 
\textcolor{Red}{
#1
}
}{}}

\setlength{\parindent}{0pt}

\begin{document}
\title{Install Guide}	

\begin{titlepage}

\begin{flushleft} \large
\leftskip0.5in

\doublespacing
\ \vspace{2em}

\ifthenelse{\vigClientBranding=1}{
\textbf{Customer:} \vigClientName\\
\textbf{Project:} \vigProjectName\\
\textbf{Attn:} \vigAttn
}

\vspace{8em}

\textbf{\Huge \vigProjectName}\\
\textbf{\Huge Installation Guide}\\
\todayYMD.VIG.\ifthenelse{\vigClientBranding=1}{\vigClientNameShort.\vigProjectNameShort.}Installation Guide.\vigDocumentRevision

\end{flushleft}

\vfill
\begin{center}
{\large \svndate}
\end{center}

\end{titlepage}

\setlength{\parskip}{0em}
\tableofcontents

\newpage

\normalsize
\setlength{\parskip}{0.5em}

\section*{Copyright}

The information provided here, including but not limited by creative, technical, promotion and information layout ideas, is property of Vide Infra Grupa SIA and is intended for the person or entity to which it is addressed. Taking of any action in reliance upon this information without prior acceptance by Vide Infra Grupa is prohibited. 

Any disclosure, copying, distribution or any action taken or omitted to be taken in reliance on it, is prohibited and may be illegal.

\copyright Vide Infra Grupa, \the\year

\vspace{2em}

Vide Infra Grupa SIA\\
Baznicas 39-5, Riga\\
LV-1010, Latvia\\
Phone: + (371) 6 729 33 90\\
Fax:  + (371) 6 729 33 45\\
e-mail: \href{mailto:info@videinfra.com}{info@videinfra.com} \\
\url{http://www.videinfra.com}

\section*{Authorship}

\begin{tabularx}{\textwidth}{X X X}
 \textbf{Name} & \textbf{Email} & \textbf{Phone} \\
 \hline\noalign{\smallskip}
% Your Name & your@email.com & +371-NNNNNNNN\\
\end{tabularx}

\section*{Revision History}
Previous document ID:

\begin{tabularx}{\textwidth}{l X l l}
 \textbf{Date} & \textbf{\mbox{Changes description}} & \textbf{Location} & \textbf{Initiated by} \\
 \hline\noalign{\smallskip}
% 2011-01-01 & Blah blah blah blah blah blah blah blah & System requirements & Your Name \\
\end{tabularx}

\newpage

\section{Introduction}

This document describes system requirements and installation procedure for {\vigProjectName}.

\subsection{Delivery Packages}

Web application is delivered in two separate packages:
\begin{itemize}
	\item \textbf{web:} This package contains web application. It is installed on Apache/PHP web server
	\item \textbf{database:} This package contains database files. It is installed on database server.
\end{itemize}

\section{System Requirements}

\subsection{Web Server}

\subsubsection{Operating System}
Web application requires one of the following operating systems:
\begin{itemize}
	\item Linux: Gentoo, SUSE, Ubuntu, RedHat, Fedora;
	\item UNIX: FreeBSD, Mac OS X.
\end{itemize}

\subsubsection{Web Server Software}
Web server: Apache HTTP Server version 2.2.x or latest from 2.2 branch. The following Apache modules must be installed:
\begin{itemize}
	\item mod\_rewrite;
	\item mod\_php;
	\item mod\_deflate.
\end{itemize}

PHP version 5.3.6 or latest from 5.3 branch must be installed, freely available at \url{http://www.php.net}. The PHP CLI and Apache HTTP Server module \textsf{mod\_php} must be installed on the Web Server as documented in \url{http://php.net/manual/en/install.unix.php}.

\note{PHP 5.3.3 required because "As of PHP 5.3.3, methods with the same name as the last element of a namespaced class name will no longer be treated as constructor. This change doesn't affect non-namespaced classes.". Some Supra classes might have methods with name equal to classname without intention to work as constructor.}
	
	\note{5.3.6 required for correct Mysql charset support on connection options without "SET NAMES" request.}

\subsubsection{PHP Extension Installation Options}

PHP must be installed or built with the following PHP extensions enabled:

\textsf{apc}, \textsf{ctype}, \textsf{curl}, \textsf{fileinfo}, \textsf{filter}, \textsf{freetype}, \textsf{gd} (with gif, jpeg, bmp, png support enabled), \textsf{hash}, \textsf{iconv}, \textsf{json}, \textsf{mbstring}, \textsf{openssl}, \textsf{pcre}, \textsf{pdo}, \textsf{pdo\_mysql}, \textsf{posix}, \textsf{session}, \textsf{simplexml}, \textsf{xml}, \textsf{zlib}.

\note{Most popular extra extensions we forget to add are \textsf{zip} (PHPExcel), \textsf{shmop} (GeoIP), \textsf{soap}, \textsf{ldap}.}

\subsubsection{PECL extension Installation}
Some extensions, like \textsf{memcache}, are not bundled with PHP and must be installed from PECL repository.

Install PECL PHP extension with command "pecl install <package name>" (more information in \url{http://php.net/manual/en/install.pecl.pear.php}) or use phpize command (more information in \url{http://php.net/manual/en/install.pecl.phpize.php})

\subsubsection{Modify PHP Configuration}

Use the php.ini-production file from bundled with PHP with values recommended for production installations.

Additionally check and modify your php.ini file according to the recommendations below:

\begin{itemize}
	\item \textsf{safe\_mode = Off}
	
	\note{Swift mailer uses "proc\_\*" functions}
	
	\item \textsf{zlib.output\_compression = On}

	\note{Will compress output automatically if browser supports it}
	
	\item \textsf{memory\_limit = 64MB}
	
	\note{Increase this for bigger projects}
	
	\item Configure these parameters depending the size of media you'll need to upload through the website:
	\\ \textsf{post\_max\_size = 10MB}
	\\ \textsf{upload\_max\_filesize = 10MB}
	
\end{itemize}

\subsubsection{Apache HTTP Server Configuration}

Configure \textsf{mod\_deflate} module to compress these content types automatically if browser accepts it:

\setlength\fboxsep{0.5em}\setbox0=\vbox{\vspace*{-\baselineskip}\begin{alltt}
AddOutputFilterByType DEFLATE text/html text/plain text/xml
AddOutputFilterByType DEFLATE text/css
AddOutputFilterByType DEFLATE application/javascript
AddOutputFilterByType DEFLATE application/rss+xml
\end{alltt}\vspace*{-\baselineskip}}\fbox{\box0}\vspace{1em}



\subsection{Database Server}
Operating system requirements are the same as for the Web Server.

\subsubsection{Database Server Software}

MySql database server 5.1.16 or latest from 5 branch must be installed. The engine and installation notes are available on \url{http://www.mysql.com}.

Please refer to MySql documentation regarding database administration concepts such as creating users and databases. The documentation is available at \url{http://dev.mysql.com/doc/}.

\subsubsection{Database Server Configuration}

Enter the MySql console and create database user \textsf{supra}.

\setlength\fboxsep{0.5em}\setbox0=\vbox{\vspace*{-\baselineskip}\begin{alltt}
  CREATE USER 'supra'@'\%' IDENTIFIED BY '<password>';
\end{alltt}\vspace*{-\baselineskip}}\fbox{\box0}

The server must be configured to accept connections from the Web Server.

\subsection{Memcached Server}
Operating system requirements are the same as for the Web Server.

\subsubsection{Memcached Server Software}

Memcached server 1.4.7 or later stable version must be installed. You can download it from \url{http://memcached.org/} and follow the instructions provided.

\subsubsection{Memcached Server Configuration}

Use the default memcached configuration. Make sure \textbf{-m} configuration parameter which limits memory usage is at least 64Mb.

The server must accept connections from the Web Server.

\subsection{Solr Server}
Operating system requirements are the same as for the Web Server.

\subsubsection{Solr Server Software}

Solr 3.5 or later from solr 3 branch must be installed.

\todo{More solr installation notes}

\subsubsection{Solr Server Configuration}

\todo{More solr configuration notes}

\section{Installation}

\subsection{Web Component Installation}

\subsubsection{Unpacking Files}

Extract \textbf{web} component's archive file. The below examples assume web application destination folder is \textsf{{\vigPathToProject}}.

\setlength\fboxsep{0.5em}\setbox0=\vbox{\vspace*{-\baselineskip}\begin{alltt}
\$ cp {\vigPackageName}-X.Y.Z.web.zip \vigPathToProject/
\$ cd \vigPathToProject/
\$ unzip {\vigPackageName}-X.Y.Z.web.zip
\$ rm {\vigPackageName}-X.Y.Z.web.zip
\end{alltt}\vspace*{-\baselineskip}}\fbox{\box0}

\note{Initial content must be provided inside web package. Uncomment if differs.}

%\ subsubsection{Copy Files}
%
%Extract \textsf{\vigPackageName-X.Y.Z.files-private.zip} archive if such file is  provided:
%
%\setlength\fboxsep{0.5em}\setbox0=\vbox{\vspace*{-\baselineskip}\begin{alltt}
%\$ cp {\vigPackageName}-X.Y.Z.files-private.zip \vigPathToProject\vigPathToSrc/files/
%\$ cd \vigPathToProject\vigPathToSrc/files/
%\$ unzip {\vigPackageName}-X.Y.Z.files-private.zip
%\$ rm {\vigPackageName}-X.Y.Z.files-private.zip
%\end{alltt}\vspace*{-\baselineskip}}\fbox{\box0}
%
%Extract \textsf{\vigPackageName-X.Y.Z.files.zip} archive if such file is provided:
%
%\setlength\fboxsep{0.5em}\setbox0=\vbox{\vspace*{-\baselineskip}\begin{alltt}
%\$ cp {\vigPackageName}-X.Y.Z.files.zip \vigPathToProject\vigPathToWebroot/files
%\$ cd \vigPathToProject\vigPathToWebroot/files
%\$ unzip {\vigPackageName}-X.Y.Z.files.zip
%\$ rm {\vigPackageName}-X.Y.Z.files.zip
%\end{alltt}\vspace*{-\baselineskip}}\fbox{\box0}

\subsubsection{Change Files and Folders Owner}

Run the below commands to set Apache user as files and folders owner assuming the Apache run user is \textsf{httpd}:

\setlength\fboxsep{0.5em}\setbox0=\vbox{\vspace*{-\baselineskip}\begin{alltt}
$ cd \vigPathToProject
$ chown -R httpd *
\end{alltt}\vspace*{-\baselineskip}}\fbox{\box0}

All files and folders under \textsf{\vigPathToProject} directory must be writable by the user \textsf{httpd}.

\subsubsection{Web Application Configuration}
Configure web application by modifying configuration file \vigPathToProject\vigPathToSrc/conf/supra.ini:

\begin{itemize}
	\item Provide database connection details;
	\item Provide \textsf{solr} server configuration;
	\item Configure other parameters according to your needs.
	%\item Any specific supra.ini settings?
\end{itemize}

\subsection{Database Component Installation}

\subsubsection{Create Database}

Enter MySql console using the MySql root user.

Create database by running the following statement:

\setlength\fboxsep{0.5em}\setbox0=\vbox{\vspace*{-\baselineskip}\begin{alltt}
  mysql> CREATE DATABASE \vigProjectNameShort CHARACTER SET UTF8;
\end{alltt}\vspace*{-\baselineskip}}\fbox{\box0}

Grant full permissions to the database for \textsf{supra} MySql user:

\setlength\fboxsep{0.5em}\setbox0=\vbox{\vspace*{-\baselineskip}\begin{alltt}
  mysql> GRANT ALL ON \vigProjectNameShort.* TO supra@'\%';
\end{alltt}\vspace*{-\baselineskip}}\fbox{\box0}

\subsubsection{Database Installation}
Extract database component archive and install database by executing the extracted SQL script file from the MySql console.

\setlength\fboxsep{0.5em}\setbox0=\vbox{\vspace*{-\baselineskip}\begin{alltt}
\$ unzip {\vigPackageName}.dump.X.Y.Z.zip
\end{alltt}\vspace*{-\baselineskip}}\fbox{\box0}

\setlength\fboxsep{0.5em}\setbox0=\vbox{\vspace*{-\baselineskip}\begin{alltt}
  mysql> use \vigProjectNameShort;
  mysql> \textbackslash. {\vigPackageName}.dump.X.Y.Z.sql
\end{alltt}\vspace*{-\baselineskip}}\fbox{\box0}

Ignore warnings about already installed "plpgsql" language.

\note{Enable search reindex section for the first time installation}
%\ section{Reindex Search}
%Rebuild of search index is required when the database is installed for the first time or when database was upgraded. It can be done in the administration interface manager "Maintenance" by pressing button "Reindex all site content". Depending on database size, rebuilding of search index may take several minutes. 

\subsection{Scheduled Task Configuration}

Application needs a scheduled task to be installed on the Web Server. The command

\setlength\fboxsep{0.5em}\setbox0=\vbox{\vspace*{-\baselineskip}\begin{alltt}
  \$ /usr/bin/php \vigPathToProject/bin/supra su:cron
\end{alltt}\vspace*{-\baselineskip}}\fbox{\box0}

must be run periodically. Recommended period is 1 minute.

Sample \textsf{crontab} configuration line:

\setlength\fboxsep{0.5em}\setbox0=\vbox{\vspace*{-\baselineskip}\begin{alltt}
* * * * * /usr/bin/php \vigPathToProject/bin/supra su:cron
\end{alltt}\vspace*{-\baselineskip}}\fbox{\box0}

\subsection{Apache HTTP Server Host Configuration}

Project may be configured in Apache HTTP Server as virtual host. This is recommended option.

Below is an example of the configuration. The actual configuration of virtual host may vary depending on system specific settings.

\setlength\fboxsep{0.5em}\setbox0=\vbox{\vspace*{-\baselineskip}\begin{alltt}
<VirtualHost *:80>
  ServerAdmin administrator@company.com
  DocumentRoot "\vigPathToProject\vigPathToWebroot"
  <Directory "\vigPathToProject\vigPathToWebroot">
    Order allow,deny
    Allow from all
  </Directory>
  ServerName www.acme.com
  ServerAlias acme.com
  DirectoryIndex index.php
  ErrorLog "/var/log/apache/acme-error\_log"
  CustomLog "/var/log/apache/acme-access\_log" common

  RewriteEngine on
  RewriteCond \%\{DOCUMENT_ROOT\}\%\{REQUEST_FILENAME\}.less -f
  RewriteRule ^(.*\textbackslash.css)\$ /cms/lib/supra/combo/combo.php?\$1 [L,NS]  
  RewriteCond \%\{DOCUMENT_ROOT\}\%\{REQUEST_FILENAME\} -f
  RewriteRule ^ - [L,NS]
  RewriteRule ^(.*)\$ /index.php\$1 [L,NS]
</VirtualHost>
\end{alltt}\vspace*{-\baselineskip}}\fbox{\box0}\vspace{1em}

Also .htaccess based configuration is possible. The Apache DocumentRoot directive must point to the folder "\vigPathToProject\vigPathToWebroot" and the .htaccess file must be placed inside it.

Example of such configuration:

\setlength\fboxsep{0.5em}\setbox0=\vbox{\vspace*{-\baselineskip}\begin{alltt}
RewriteEngine on
RewriteCond \%\{REQUEST_FILENAME\}.less -f
RewriteRule ^(.*\textbackslash.css)\$ /cms/lib/supra/combo/combo.php?\$1 [L,NS]
RewriteCond \%\{REQUEST_FILENAME\} -f
RewriteRule ^(.*)\$ - [L,NS]
RewriteRule ^(.*)\$ index.php/\$1 [L,NS]
\end{alltt}\vspace*{-\baselineskip}}\fbox{\box0}

\section{Upgrade Procedure}

\subsection{Overview}
Product upgrade process is similar to product installation. Always backup content folders and database prior to upgrade. Always stop Apache service prior to product upgrade.

\subsection{Web Component Upgrade}

\subsubsection{Stop Web Server}
Shut down Apache web server by running the following command from shell:

\setlength\fboxsep{0.5em}\setbox0=\vbox{\vspace*{-\baselineskip}\begin{alltt}
\$ apachectl stop
\end{alltt}\vspace*{-\baselineskip}}\fbox{\box0}

\subsubsection{Backup Web Application}
Backup your current product web component:

\setlength\fboxsep{0.5em}\setbox0=\vbox{\vspace*{-\baselineskip}\begin{alltt}
\$ mv \vigPathToProject \vigPathToProject-1.0
\end{alltt}\vspace*{-\baselineskip}}\fbox{\box0}

\subsubsection{Install Web Application}
Extract \textbf{web} package in your product installation directory:

\setlength\fboxsep{0.5em}\setbox0=\vbox{\vspace*{-\baselineskip}\begin{alltt}
\$ mkdir \vigPathToProject
\$ unzip {\vigPackageName}-X.Y.Z.web.zip -d \vigPathToProject
\end{alltt}\vspace*{-\baselineskip}}\fbox{\box0}

\subsubsection{Configure Web Application}
Edit \textbf{conf/supra.ini} file. You may copy individual lines/settings from the old version of supra.ini.

\textbf{Note:} do not copy entire supra.ini from the old version.

\subsubsection{Copy Content from backup}
Copy content folder from the backup:

\setlength\fboxsep{0.5em}\setbox0=\vbox{\vspace*{-\baselineskip}\begin{alltt}
\$ rm -Rf \vigPathToProject/content
\$ cp -Rf \vigPathToProject-1.0/content \vigPathToProject/content
\$ rm -Rf \vigPathToProject/webroot/content
\end{alltt}\vspace*{-\baselineskip}}\fbox{\box0}

\subsubsection{Change Files and Folders Owner}
Run the below commands to set apache user as files and folders owner:

\setlength\fboxsep{0.5em}\setbox0=\vbox{\vspace*{-\baselineskip}\begin{alltt}
\$ cd \vigPathToProject
\$ chown -R apache *
\end{alltt}\vspace*{-\baselineskip}}\fbox{\box0}

\subsubsection{Restart Web Server}
Clean up cache and restart Apache:

\setlength\fboxsep{0.5em}\setbox0=\vbox{\vspace*{-\baselineskip}\begin{alltt}
\$ cd \vigPathToProject/data
\$ rm -rf *
\$ apachectl restart
\end{alltt}\vspace*{-\baselineskip}}\fbox{\box0}

\subsection{Database Component Upgrade}
To upgrade the database component follow the next steps:

\begin{enumerate}
	\item Copy database package {\vigPackageName}-X.Y.Z.database.zip to database server.
	\item Extract the archive.
	\item Perform database backup as postgres user.
	\item Run all *.sql files from the database package. Make sure you are executing them in the right order -- each SQL script has its own ordinal number. Execute the scripts in ascending order. Be sure scripts return no errors.
\end{enumerate}

\end{document}
